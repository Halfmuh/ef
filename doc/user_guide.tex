\section{Руководство пользователя}

\subsection{ Особенности реализации PIC в программе \prognv }

Epifc-v0.01 поддерживает только прямоугольную расчетную область. 
При дискретизации области используется прямоугольная сетка с фиксированным шагом.
\todo{ картинка }


\todo{Генерация частиц.}
Пока есть возможность задать только один источник.
Кроме того, генерация новых частиц во время работы тоже пока не поддерживается.

\todo{Связь с концентрацией реальных частиц и суммарным зарядом в области.}
У каждой макрочастицы есть заряд и масса. 
Если концентрация реальных частиц
массой $m_{real}$ и зарядом $q_{real}$
в объеме $V$ равна $n_0\, [1/cm^2]$, 
то масса $m$ и заряд $q$ каждой макрочастицы определяются соотношениями
\begin{gather}
  m = m_{real} \frac{ n_0 V }{ N },
  \quad
  q = q_{real} \frac{ n_0 V }{ N },
\end{gather}
где $N$ - число макрочастиц.
Пользователю необходимо задать массу $m$, заряд $q$ и число макрочастиц $N$
для каждого источника \todo{( см. следующий раздел )}.

\todo{Распределения}
По координатам - равномерно в прямоугольнике.
По импульсам - максвелловское распределение.
\begin{gather}
  f( \vec{p} ) dp = \frac{1}{ 2 \pi m \theta } e^{ - \dfrac{ p_x^2 + p_y^2 }{ 2 m \theta } }
\end{gather}
где $\theta = k T$, $k$ - постоянная Больцмана, $T$ - температура.
В конфигурационном фаиле необходимо задать $\theta$ для каждого источника.


После инициализации частиц запускается алгоритм PIC. 
Вначале происходит интерполяция зарядов макрочастиц на узлы сетки.
Предполагается, что частицы вносят вклад только в ближайшие узлы. 
\todo{Форма частиц - ступенька ( поискать официальное название для этого ).}
\todo{Вставить картинку.}

\todo{Здесь оставить только общие уравнения. Дискретные убрать в раздел с описанием программы.}

Следующий шаг - расчет потенциалов. 
Epicf-v0.01 не поддерживает магнитное поле
и использует только электрическое ( электростатическая модель, ES-PIC ).
Для его нахождения нужно решить уравнение Пуассона:
\begin{gather}
  \Delta \phi = - 4 \pi \rho
\end{gather}
Поддерживаются граничные условия только 1 рода ( на потенциал, а не на поле ).
Они задаются в конфигурационном фаиле.

По найденным значениям потенциала расчитывается значение поля в узлах решетки:
\begin{gather}
  \vec{ E } = - \nabla \phi
\end{gather}

Затем обновляются импульсы и координаты частиц
( при этом происходит обратная интерполяция полей из узлов решетки на частицы ).
\begin{gather}
  \frac{ d \vec{p} }{ d t } = \vec{ F } = q \vec{ E }
  \\
  \frac{ d \vec{r} }{ d t } = \frac{ \vec{p} }{ m }
\end{gather}
Используется схема Leap-frog ( см. след раздел ).
\todo{ пока еще не используется. надо доделать сдвиг на dt/2 }

Запись в фаил. 
В конце временного шага можно сохранить результаты расчетов в фаил. 
Записывается вся информация, необходимая для возобновления расчетов.
Детально формат фаила описан в \todo{следующем} разделе. 
Имя выходного фаила генерируется по параметрам в конфигурационном фаиле.
Номер временного шага для записи тоже берется из конфига.




\subsection{ Параметры конфигурационного фаила }

Для пользователя, взаимодействие с программой должно ограничиваться изменениями в конфигурационном фаиле.

\todo{Выбор согласованных единиц измерения - задача пользователя. Написать про размерности.}

Описание параметров:

\subsection{ Визуализация результатов }

В комплекте с программой идет скрипт для визуализации результатов.
( скрипт на R - см. раздел ``установка'' ).
Лежит в директории plot. Вызывается командой \todo{пример}.

Пока доступно 3 опции: 
\begin{itemize}
\item -p, --potential - строит значения потенциала в узлах сетки.
\item -d, --density - строит значения плотности заряда в узлах сетки.
\item -P, --particles - рисует частицы и направления их импульсов.
\end{itemize}
\todo{Вставить пару примеров картинок.}

\subsection{ Установка }
\todo{Тоже написать пару слов.}
%%% Local Variables: 
%%% mode: latex
%%% TeX-master: "major"
%%% End: 
