\section{ Планы }
Вначале надо доделать всякие мелочи для v0.01.
\begin{itemize}
\item Документация
\item \sout{Расставить правильные коэффициенты в уравнениях.}
\item \sout{Сдвиг на t/2 для leap-frog.}
\item \sout{Путь к конфиг-фаилу из командной строки.}
\end{itemize}

Планы для v0.02:
\begin{itemize}
\item Перевести документацию на английский.
\item Перейти на конечные элементы для уравнения Пуассона.
  Пока оставить прямоугольную область. 
  Использовать deal.ii как библиотеку для конечных элементов.
\item Возможно, переписать какие-то части на C++. Хотя это можно отложить на потом.
\end{itemize}


В обозримом будущем нужно добавить:
\begin{itemize}
\item Диагностика: проверка сохранения энергии и импульса.
\item Источники: поддержка нескольких источников, генерация новых частиц во время работы программы.
\end{itemize}

Глобальные планы.
\begin{itemize}
\item Конечные элементы.
\item Магнитное поле.
\item Распараллеливание.
\item Интеграция 2d версии с какой-нибудь программой для моделирования.
\item 3d-версия.
\item Интеграция 3d версии с какой-нибудь программой для моделирования.
\end{itemize}


%%% Local Variables: 
%%% mode: latex
%%% TeX-master: "major"
%%% End: 
