\section{ Планы }
v0.01.
\begin{itemize}
\item Документация
\item \sout{Расставить правильные коэффициенты в уравнениях.}
\item \sout{Сдвиг на t/2 для leap-frog.}
\item \sout{Путь к конфиг-фаилу из командной строки.}
\end{itemize}

v0.02:
\begin{itemize}
\item Документация
\item \sout{Источники: поддержка нескольких источников, генерация новых частиц во время работы программы.}
\end{itemize}

v0.03:
\begin{itemize}
\item Документация
\item \sout{Возможно, переписать какие-то части на C++. Хотя это можно отложить на потом.}
\end{itemize}

v0.04
\begin{itemize}
\item Документация
\item Трехмерная область с помощью конечных разностей.
\end{itemize}

\bigskip \bigskip
В обозримом будущем:
\begin{itemize}
\item Перевод с GSL на PETSc.
\item Магнитное поле.
\item Подумать о шаблонизации 2д и 3д кода.
\item Поддержка прямоугольных препятствий внутри расчетной области.
\item Диагностика: проверка сохранения энергии и импульса.
\end{itemize}

\bigskip \bigskip
Глобальные планы:
\begin{itemize}
\item Документация на английском.
\item Перейти на конечные элементы для уравнения Пуассона.
  Использовать deal.ii как библиотеку для конечных элементов.
\item Распараллеливание.
\item Интеграция с какой-нибудь программой для моделирования.
\end{itemize}


%%% Local Variables: 
%%% mode: latex
%%% TeX-master: "major"
%%% End: 
