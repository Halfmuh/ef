\section{ Установка }
Исходные коды Epicf доступны на github. 
Проще всего загрузить их оттуда с помощью соответствующей команды git
\begin{verbatim}
git clone https://github.com/noooway/epicf
\end{verbatim}

Для сборки понадобится компилятор С++ (например, входящий в состав GCC или Clang)
а также библиотеки \href{http://www.gnu.org/software/gsl/}{GSL} и 
\href{http://www.boost.org/}{Boost}.
В большинстве дистрибутивов GNU/Linux эти программы есть в стандартных репозиториях.
На примере Debian и Ubuntu, команды для установки имеют вид:
\begin{verbatim}
apt-get install build-essential 
apt-get install libgsl0-dev
apt-get install libboost-all-dev
\end{verbatim}
При установке библиотек нужны именно development-версии,
поскольку заголовочные фаилы поставляются только в них.

Когда все необходимое установлено, для сборки Epicf достаточно 
перейти в директорию с исходными кодами и выполнить команду make
\begin{verbatim}
cd /some-path/epicf
make
\end{verbatim}

Если компиляция прошла успешно, то в директории должен появиться исполняемый фаил epicf.out.
Запустить программу с выбранным конфигурационным фаилом можно с помощью 
\begin{verbatim}
./epicf.out test.conf
\end{verbatim}

В комплекте с программой идет скрипт для визуализации результатов.
Скрипт написан на языке R, поэтому чтобы им воспользоваться, нужно установить R:
\begin{verbatim}
apt-get install r-base-dev
\end{verbatim}
Также для работы скрипта требуется установить пакет 'optparse' для R.
Это можно сделать, выполнив в командной строке
\begin{verbatim}
R -e "install.packages('optparse', repos='http://cran.us.r-project.org')"
\end{verbatim}
или командой
\begin{verbatim}
install.packages('optparse')
\end{verbatim}
из консоли R.

%%% Local Variables:
%%% mode: latex
%%% TeX-master: "epicf"
%%% End:
