\section{Общее введение}
\todo{ Описание метода частиц в ячейке }
Метод частиц в ячейке ( particle in cell, PIC ) используется для 
численного моделирования систем заряженных частиц. 
Основные области применения - физика плазмы и ускорительная техника.

Основные идеи:
Реальные частицы заменяются на некоторое количество макрочастиц.
Время дискретизуют.
На каждом временном шаге решают систему уравнений на потенциалы. 
Потом вычисляют электрическое и магнитное поле.
Потом эти значения полей используют для вычисления силы, действующей на каждую макрочастицу и изменения ее импульса и координаты. 

Детальное описание можно найти в [ссылки]

\todo{Какие PIС программы есть. Обзор. Чем не устраивают.}
Есть несколько свободных программ, реализующих PIC:

\begin{itemize}
\item XOOPIC \url{http://ptsg.eecs.berkeley.edu/#Software}
\item PIConGPU \url{http://www.hzdr.de/db/Cms?pOid=31887&pNid=0}
\item PicUp3D \url{http://dev.spis.org/projects/spine/home/picup}
\end{itemize}

\todo{ Цели epicf и отличие от других программ. }
Основная цель - интеграция с системой проектирования или 3д-моделирования.
Нерелятивистские задачи.



%%% Local Variables: 
%%% mode: latex
%%% TeX-master: "major"
%%% End: 
