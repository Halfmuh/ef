\section{Общее введение}
\textbf{ Описание метода частиц в ячейке. }
Метод частиц в ячейке ( particle in cell, PIC ) используется для 
численного моделирования систем заряженных частиц. 
Основные области применения - физика плазмы и ускорительная техника.

В-принципе, поведение системы большого числа заряженных 
частиц можно смоделировать решая численно уравнения Ньютона для каждой частицы.
При этом сила, действующая на частицу, будет суммой всех сил,
действующей между заданной частицей и всеми остальными частицами системы
\begin{gather}
  \frac{ d \vec{p}_i }{ dt } = \vec{F} = \sum_{j \neq i} \vec{F}_{ij}.
\end{gather}
На практике, однако, такой подход нереализуем. Основная проблема - слишком
большой объем требуемых вычислений. 
В системе из $N$ частиц число возможных пар - $N^2/2$.
При типичных $N \approx 10^{?15-23?}$ эти неприемлимо много.

В основе метода PIC лежат две упрощающих идеи. 
Во-первых, поведение большого числа реальных частиц можно 
смоделировать, заменив их на сравнительно небольшое число макрочастиц. 
Это позволяет уменьшить требуемую вычислительную нагрузку, однако все еще
недостаточно, чтобы такой подход можно было использовать на практике.
(число частиц, которое которое удается моделировать таким образом порядка \todo{сделать программу, посмотреть} ).
Вторая идея - вместо того, чтобы считать взаимодействие между отдельными 
частицами, в PIC считается электрическое и магнитное поле, которое
создают все частицы системы. После чего сила, действующая на каждую отдельную
частицу, определеятся из этого поля. 

Для нахождения полей приходится прибегать к решению уравнений Максвелла.
Это тоже приходится делать численно. 

Детальное описание можно найти в [ссылки]

\textbf{Обзор существующих PIС-программ.}
Есть несколько свободных программ, реализующих PIC:

\begin{itemize}
\item XOOPIC \url{http://ptsg.eecs.berkeley.edu/#Software}
\item PIConGPU \url{http://www.hzdr.de/db/Cms?pOid=31887&pNid=0}
\item PicUp3D \url{http://dev.spis.org/projects/spine/home/picup}
\end{itemize}

\todo{ Цели epicf и отличие от других программ. }
В отличие от приведенных выше программ Epicf ориентирован на нерелятивистские задачи.
Также основной целью Epicf является моделирование произвольных геометрий,
чего можно достичь с помощью интеграции с системой проектирования или 3д-моделирования.

%%% Local Variables:
%%% mode: latex
%%% TeX-master: "epicf"
%%% End:
